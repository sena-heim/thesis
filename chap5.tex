\chapter{結論}\label{sec_conc}
本研究では受容やの大きさや形状が可変な deformable convolution が物体によって受容野の形状をどのように変更しているのかを受容やを可視化することにより視覚的に評価した. その結果, 受容野が学習できる範囲外までの広がっていると考えた. そこで受容野の広がる範囲をoffsetを算出する特徴マップが表現できる範囲内に抑えるために新たな活性化関数を提案した. 提案した活性化関数をMIoU, IoU で評価するのに加え, 受容野を可視化して意味のある変形をしているかを確認した. 結果は, MIoUが0.3pt向上し, IoUは多くのクラスにおいて精度の向上をパラメータの追加をせずに達成した. よってdeformable convolutionの受容野を制限することで学習が促進すると結論付けられる.