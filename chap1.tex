\begin{chapter}{序論}
    近年の少子高齢化による労働人口の減少により, 製造業界や物流業界では十分に労働者を確保することが困難になっている. 人手不足を解決するために, ロボットなどを用いることで, 人が行っている作業を自動化させることが取り組まれている. 自動化の例として, 倉庫での倉庫での荷積みや荷下ろし, ピッキング作業などが挙げられる. これらの作業はものを掴む, 運ぶ, 離す動作が必要であり, 対象の物体の正確な位置と形状を把握できないと対象の物体や周辺の積荷などを破損してしまう可能性がある. 現在までに研究されていた``物体検出''\cite{rcnn}\cite{yolo}は画像中にある物体のクラスと位置を認識して, 短形の枠で囲む技術であった. そのため, 大まかな物体の位置は把握できても物体の形状までは認識できていなかった. この問題を解決するために``Semantic Segmentation''\cite{semaseg}\cite{semaseg2}が登場した. \\
    Semantic Segmentationは入力画像に対してピクセル単位でクラス分類を行うことで, 物体の輪郭で領域分割する技術である. これにより画像中の物体のクラス, 位置のみならず, 詳細な形状まで認識することができるため, Semantic Segmentationはロボットビジョンへの応用が期待されている. このような分野に需要があるため, 画像認識分野においてSemantic Segmentationに関する研究が盛んに行われている.


    % 近年, 製造業界や物流業界では人手がロボットに置き換わってきている. これは人口全体に対しての働き手人口の不足による影響が大きい. 例えば,  

    
\end{chapter}
  