\documentclass[12pt,a4j]{ronbun}

% 必要に応じて使うパッケージを書きましょう
\usepackage{epsfig}
\usepackage{url}		% URL参考文献を引くため
\usepackage{amsmath}	% キャプション内部での改行用
% \usepackage[dvips]{graphicx}
% \usepackage[]{graphicx}
%\usepackage[dvipdfmx]{}
% \usepackage{multicol}	% 表で横結合
\usepackage{multirow}	% 表で縦結合
% \usepackage{sty/slashbox}   % 表内部でスラッシュ
% \usepackage{columncolor}
% \usepackage{hspace}
%\usepackage{amssymb}
%\usepackage{hangcaption}
%\usepackage{tabularx}
%\usepackage{graphicx}
%\usepackage{amsmath}
%\usepackage{here}
%\usepackage{epsbox}
% \usepackage{color}
% \usepackage{ascmac}
\usepackage{lscape}

\usepackage{here}
%\usepackage{epsbox}
\usepackage[subrefformat=parens]{subcaption}


\textwidth=160mm

%--------------------------------------------------
\begin{document}
	\bibliographystyle{junsrt}

	\begin{titlepage}

	\begin{flushright}
		\begin{tabular}{|c|}
			\hline
			指導教員印	\\ \hline
			\\
			\\
			\\ \hline
		\end{tabular}
	\end{flushright}

	\begin{center}

		\vspace{1cm}

		\huge

		2021年度 卒業論文

		\vspace{2cm}

		特徴量の抽象度に応じて受容野の範囲に\\
		制限を設けたDeformable convolution

		\vspace{7cm}

	\end{center}

	\Large

	\begin{flushright}
		\begin{tabular}{ll}
			指導教員	&荒井 秀一 教授					\\ \\
			\multicolumn{2}{l}{東京都市大学 知識工学部}	\\
			\multicolumn{2}{l}{情報科学科}				\\
			\multicolumn{2}{l}{知識情報処理研究室}		\\
			1812035		&折田 汐凪					\\
		\end{tabular}
	\end{flushright}

\end{titlepage}

	\tableofcontents
	% チャプターが増えるごとに'chap[N].texファイルを作成してここに追記しましょう。
	\begin{chapter}{序論}
    近年の少子高齢化による労働人口の減少により, 製造業界や物流業界では十分に労働者を確保することが困難になっている. 人手不足を解決するために, ロボットなどを用いることで, 人が行っている作業を自動化させることが取り組まれている. 自動化の例として, 倉庫での倉庫での荷積みや荷下ろし, ピッキング作業などが挙げられる. これらの作業はものを掴む, 運ぶ, 離す動作が必要であり, 対象の物体の正確な位置と形状を把握できないと対象の物体や周辺の積荷などを破損してしまう可能性がある. 現在までに研究されていた``物体検出''\cite{rcnn}\cite{yolo}は画像中にある物体のクラスと位置を認識して, 短形の枠で囲む技術であった. そのため, 大まかな物体の位置は把握できても物体の形状までは認識できていなかった. この問題を解決するために``Semantic Segmentation''\cite{semaseg}\cite{semaseg2}が登場した. \\
    Semantic Segmentationは入力画像に対してピクセル単位でクラス分類を行うことで, 物体の輪郭で領域分割する技術である. これにより画像中の物体のクラス, 位置のみならず, 詳細な形状まで認識することができるため, Semantic Segmentationはロボットビジョンへの応用が期待されている. このような分野に需要があるため, 画像認識分野においてSemantic Segmentationに関する研究が盛んに行われている.


    % 近年, 製造業界や物流業界では人手がロボットに置き換わってきている. これは人口全体に対しての働き手人口の不足による影響が大きい. 例えば,  

    
\end{chapter}
  		% 序論
	\chapter{背景}\label{chap:haikei}
Semantic segmentataionでは, 深層学習を用いた手法が多く提案されている. その中でも CNN(Convolutional Neural Network)\cite{cnn}を用いた手法により認識精度は飛躍的に向上した. そのことから, 今までにCNNをもとにした手法は数多く提案され, アーキテクチャが改良されてきた. そして近年, convolutionを改良したdeformable convolution\cite{defconv}を用いたことにより, さらに認識精度が向上した. しかし, deformacle convolutionの挙動は調査されていない. \\
本章では, まず\ref{chap:semantic}節でsemantic segmentationを概説することにより本研究分野の立ち位置を明確にする. その上で, \ref{sec:zyuurai}節でsemantic segmentationにおいて現在までに提案されてきたCNNをもとにした従来手法を紹介し, 従来手法のネットワークに共通する構造上の問題点を明らかにする. 次に, \ref{subsec:hrnet}項で従来手法の問題点を改善した手法を取り上げ, 最後に\ref{sec:mokuteki}節でこれらの従来手法の問題点を明らかにしたうえで, 本論文の研究目的を定義する.


\section{Semantic segmentation}\label{chap:semantic}
Semantic segmentationは入力画像に対してピクセル単位でクラス分類を行うことで, 画像中の様々な物体をそれぞれの外形で領域分割する技術である[3][4]. 図\ref{fig:seg_rei}中の上段は入力画像であり, 下段に示すクラスごとに入力画像を分離した結果が中段の画像である. そのため, semantic segmentation は画像中に存在する様々な物体が,それぞれどのクラスに属し,どのような形状で,どの位置に存在するのかを表現可能であり,シーン理解のタスクにおいて重要な手法である.近年では, 深層学習を用いた手法の提案により, 以前よりさらに高精度な推論が可能になった\cite{long2015fully}. 



\begin{figure}[H]
    \centering
    \includegraphics[scale=0.28]{./images/fig1.eps}
    \caption{Semantic Segmentatioinの例(上段:入力画像, 中段:分類結果, 下段:クラス凡例)\cite{semaseg2}}
    \label{fig:seg_rei}
\end{figure}

\section{従来研究}\label{sec:zyuurai}
Semantic segmentationは深層学習を用いた手法が多く提案されてきた. それらの多くはCNNを元としている. 本節では, 多くの手法の元となっているCNNを説明した後に, CNNを用いた手法の代表例である Bottleneck 構造の CNNを紹介する.  その後に従来のBottleneck 構造の CNN より高精度なセグメンテーションを可能にした HRNet (High-Resolution Network)\cite{hrnet}を説明した上で,  CNNに変更を加えることでHRNetを改良した DHRNet (Deformable High-Resolution Network)\cite{dhrnet} を説明する.

\subsection{CNN(Convolutional Neural Network)}\label{subsec:cnn}
画像認識では主にCNN\cite{cnn}が用いられてきた. CNNは図\ref{fig:conv_ex}に示すような深層学習に用いられる多層ニューラルネットワークの1つである. 図\ref{fig:conv_ex}では入力画像を畳み込み層(convolution layer), プーリング層(pooling layer)で特徴量抽出した後, 一般的な全結合ネットワークがもつ全結合層(full connection layer)に通すことでクラス分類をおこなっている. ネットワークを多層にすることで,入力に近い層ではエッジなどの特徴を抽出し, 深い層になるにつれ, より抽象的な特徴を抽出できるように構成されている. semantic segmentationでは, ネットワークの出力がCNNの全結合層を畳み込み層に置き換えた, 畳み込み層, プーリング層 のみで構成される FCN(Fully Convolutional Network)\cite{long2015fully} をベースとしたアーキテクチャが多く提案されている. 
\begin{figure}[H]
  \centering
  \includegraphics[scale=0.35]{./images/example_cnn2.eps}
  \caption{Convolutional Neural Networkの例}
  \label{fig:conv_ex}
\end{figure}
\subsubsection*{畳み込み(Convolution)}
% この処理により画像の局所的な特徴を抽出でき, 物体の形状を学習できる.
畳み込みは, 入力データと学習した重みを持つフィルタとで畳み込み演算をして, ``特徴マップ''を出力する処理である. 畳み込み演算は, 図\ref{fig:conv}の灰色で表したフィルタを一定間隔でスライドさせながら, フィルタの要素と入力データの対応する要素を乗算し, その和を求める演算である. また畳み込みはフィルタの数, フィルタのサイズ, ストライド, パディングという設定する必要のあるパラメタを持つ. \\
まず, フィルタの数があり, 設定したフィルタの数だけ特徴マップを出力する. \\
次にフィルタのサイズはフィルタの幅と高さであり, 図\ref{fig:conv}のカーネルは高さ3, 幅3である. 
\begin{figure}[H]
    \centering
    \includegraphics[scale=0.35]{./images/fig2.eps}
    \caption{Convolutionの例(*は畳み込み積分)}
    \label{fig:conv}
\end{figure}

入力データを$I(x, y)$, 出力特徴マップを$O(x, y)$, フィルタカーネルを$h(m, n)$, フィルタサイズを$(2M+1)\times(2N+1)$とした場合, 畳み込みは式(\ref{eq:conv1})のように書ける.

\begin{equation}
  O(x, y) = \sum_{m=-M}^M \sum_{n=-N}^N h(m,n)I(x+m, y+n) \label{eq:conv1}
\end{equation}

入力データ$I(x, y)$のサイズを$(X, Y)$とすると, 出力特徴マップ$O(x, y)$は($M<=x<X-M, N<=y<Y-N$)の範囲までしか求められず, 出力特徴マップのサイズが小さくなってしまう. これを防ぐためにはパディングという処理を行う. パディングは, 図\ref{fig:padding}のように入力データの周囲を一定の値で埋める処理である. パディングにより出力特徴マップ$O(x, y)$は($0<=x<X, 0<=y<Y$)のように入力データと同じサイズの出力特徴マップになる.  
\begin{figure}[H]
  \centering
  \includegraphics[scale=0.35]{./images/padding.eps}
  \caption{パディングの例}
  \label{fig:padding}
\end{figure}

ストライドはフィルタカーネルを移動させる間隔を決めるパラメータである. パディングが入力データと出力特徴マップのサイズを合わせるための処理であるのに対し, ストライドは1より大きい整数に変更することで出力特徴マップのサイズを縮小させることができる.  

ストライドを1より大きい整数$s$に変更したConvolutionは式(\ref{eq:stconv})のように書ける.

\begin{equation}
 O(x, y) = \sum_{m=-M}^M \sum_{n=-N}^N h(m,n)I(sx+m, sy+n) \label{eq:stconv}
\end{equation}

これにより$O(x,y)$は($0<=x<\frac{X}{s}, 0<=y<\frac{Y}{s}$)のように出力特徴マップのサイズが小さくなる. しかし$s$を大きくしていくと, それに伴い特徴を見落とす可能性が高くなる. 特に$s$がフィルタカーネルの高さ, または幅を超えてしまうと明らかにフィルタに覆われない領域が出現し, 特徴を見落としてしまう. そのため, 特徴を見落とさないためには, ストライド$s$をフィルタカーネルの高さ, 幅より小さいサイズに収めるか, 後述するプーリングにより出力特徴マップを小さくする方法がある. 



\subsubsection*{プーリング(Pooling)}
プーリングは, 畳み込みにより得られた特徴マップのサイズを縮小する処理である. この処理はパラメータの削減や, 受容野を広げる役割があり, また微小な位置ずれに対しロバストになる. 領域内の最大値を得るMax Pooling(図\ref{fig:pooling})や領域内の平均を得るAverage Poolingなどがある. ストライド2の畳み込みとストライド1の畳み込み + $2\times2$Poolingは出力特徴マップのサイズという面では同じであるが, 前夜が畳み込みの間隔を大きくして特徴マップを小さくしただけなのに対し, 後者は特徴量抽出を行なった後により値が大きい特徴を残して抽象化するという面で, 後者の方が処理の意味付けが明確にできる.

\begin{figure}[H]
  \centering
  \includegraphics[scale=0.35]{./images/fig3.eps}
  \caption{Max Poolingの例}
  \label{fig:pooling}
\end{figure}

\subsection{Bottleneck構造のCNN}\label{subsec:en_de_cnn}
Semantic segmentation の手法として, bottleneck構造のCNNが多く用いられている. 
bottleneck構造のCNNは, 入力画像を特徴量抽出しつつ, 解像度を下げていき, 学習可能な層などを用いて入力画像と同等の解像度まで復元する構造をもっている.
bottleneck構造のCNNの例を図\ref{fig:segnet}に示す. 図\ref{fig:segnet}のエンコーダ部分では特徴量を抽象化していき, 低解像度になった特徴マップを図\ref{fig:segnet}のデコーダー部分で特徴抽出をしつつ, 高解像度化している. 
\begin{figure}[H]
    \centering
    \includegraphics[scale=0.22]{./images/fig4.eps}
    \caption{bottleneck構造の CNN(SegNet\cite{segnet})}
    \label{fig:segnet}
\end{figure}

bottleneck構造のCNNの代表的なアーキテクチャには, ``SegNet''\cite{segnet}, ``U-Net''\cite{unet}, ``RefineNet''\cite{refinenet}などがある.
これらの手法では, 入力に近い部分で抽出した特徴マップを一旦抽象化してしまうため, 高解像度の特徴マップがもつ物体の詳細な形状の情報が失われてしまう. これは物体の形状を正確に認識する必要があるsemantic segmentationにおいて大きな問題である.

\subsection{HRNet (High-Resolution Network)}\label{subsec:hrnet}
HRNet\cite{hrnet}は, 高解像度と低解像度の特徴マップを直列に扱う従来の手法とは異なり, ネットワーク全体を通し高解像度の特徴マップを維持できる構造を持っている. 高解像度の特徴マップのみでは抽象的な特徴は抽出できないため, 層が深くなるごとに低解像度化したネットワークを追加することで, 図\ref{fig:hrnet}に示す最上段の灰色にあたる高解像度部分から最下段の赤色にあたる低解像度までの特徴を並列に処理できる構造になっている. この構造によりHRNetは, bottleneck構造のCNNの構造上の問題である, 物体の細かな形状の情報を維持できない点を改善した. 

\begin{figure}[H]
    \centering
    \includegraphics[scale=0.27]{./images/hrnet_fig.eps}
    \caption{HRNet}
    \label{fig:hrnet}
\end{figure}


\subsection{DHRNet (Deformable convolution High-Resolution Network)}\label{subsec:dhrnet}
DHRNet\cite{dhrnet}は HRNet の低解像度部分にある畳み込み層を受容野が変形可能な deformable convolution に置き換えたものである. 
前項までにsemantic segmentataionのCNNを用いた代表的な手法を紹介いてきたが, 根本となる畳み込み層を改善する試みはされていなかった. 畳み込みは\ref{subsec:cnn}項で述べたように, 入力画像に重みを持った固定サイズのフィルタカーネルをかけ特徴を抽出する処理である. そのため, 入力特徴マップのうちある出力特徴量に影響を及ぼす範囲である受容野は固定である. 
しかし, semantic segmentationでは, 様々な形状や大きさの物体の特徴を抽出する必要があるため, 固定形状の受容野では特徴抽出に適切ではない. この問題を解決するために, DHRNetはdeformable convolutionという畳み込みをHRNetに適応した\cite{dhrnet}. deformable convolutionは, 追加の畳み込み層で学習したoffsetベクトルを用いることで, 受容野を可変形状にする畳み込み層である.  図\ref{fig:offset}に受容野を可変にしたフィルターカーネルの例を示す. 図\ref{fig:offset}の赤点が元の畳み込みの受容野, 赤線はoffsetベクトル, そして青点がoffsetにより移動した後の受容野を表している.受容野がoffsetベクトルによって可変になっていることがわかる.DHRNetは都市シーンのセグメンテーションのタスクなどで優秀な成績を収めている. 
この手法に関する詳しい説明は\ref{chap:teian}章で述べる.

\begin{figure}[H]
    \centering
    \includegraphics[scale=2]{./images/offset.eps}
    \caption{offsetによる受容野の変化例}
    \label{fig:offset}
\end{figure}

\section{研究目的}\label{sec:mokuteki}
\ref{subsec:dhrnet}節で紹介したDHRNetは, HRNetと比較して識別精度を向上させるさせることに成功した. しかし, なぜdeformable convolutionを導入したことで認識精度が向上したのかは明らかにされていない. そこで, deformable convolutionの挙動を明らかにするために, deformable convolutionにより変化したフィルタカーネルの位置を可視化し受容野の変化を視覚的に確認した.
% 目的である ``物体の形状や大きさにより受容野を変更することで, 各物体に応じたより適切な特徴量抽出を実現する'' ことが達成できたかどうかの検討は行われていない. 
% そこで, deformable convolution により変化したフィルタカーネルの位置を可視化することにより, 受容野の変化を視覚的に確認した. 
DHRNetにはdeformable convolution層が2層含まれており, 図\ref{fig:defconv_2_offset}(a)は1層目, 図\ref{fig:defconv_2_offset}(b)は2層目の可視化結果である. 可視化にあたってofsetの変化を評価するために7つの画素を選んだ. 前提として, offsetの適切な変化量を定義することは困難であるため, 受容野が必ず異なる形状であるべきだと考えた評価点を選定している. 仮説としては, エッジ付近と物体の中心付近の受容野の形状には明確な違いが現れると考えたため, 評価の対象とする画素は, 道路と歩道とのエッジ付近や, 物体の中心付近とした. \\
% 可視化に選んだある画素を中心とする従来の3x3サンプリング点と変更後のサンプリング点とを表示する. 
\begin{figure}[h]
    \begin{minipage}{0.5\hsize}
     \begin{center}
      \includegraphics[scale=0.10]{./images/offset_deform22.eps}
      \subcaption{1層目のdeformable convolution の offset}
       \label{fig:defconv2_1_offset}
      \end{center}
    \end{minipage}
    \begin{minipage}{0.5\hsize}
     \begin{center}
      \includegraphics[scale=0.10]{./images/offset_deform24.eps}
      \subcaption{2層目のdeformable convolution の offset}
      \label{fig:defconv2_2_offset}
      \end{center}
    \end{minipage}
    \caption{deforamble convolutionによる受容野の変化を可視化}
\end{figure} \label{fig:defconv_2_offset}
可視化結果である図\ref{fig:defconv_2_offset}(a)と 図\ref{fig:defconv_2_offset}(b)を見ると受容野の大きさが変化していたが, 評価画素ごとに受容野の形状を比較すると, 変化が乏しいことが確認できた. さらに, 図\ref{fig:defconv_2_offset}の各サンプリング点の左右方向のoffsetの大きさに注目すると, 明らかにoffsetベクトルが大きいものが存在する. \\
畳み込みは重みを持ったフィルタカーネルと, それと同じサイズの入力特徴量とを掛け合わせる処理であるため, 出力特徴量が表現できるのは畳み込まれた局所的な特徴量の範囲内に制限される. つまり畳み込みは限られた範囲の中で特徴抽出を繰り返す処理である. ここで, deformable convolutionのoffsetは畳み込み層の出力特徴量であるため, 畳み込み層の入力特徴量が持つ情報の範囲内のみ表現可能である.
そのため, deforamble convolutionの受容野の範囲は入力特徴量がもつ情報の範囲に制限されるべきである. しかし, 理由は後述するが, 現状のdeformable convolutionのアーキテクチャは受容野の範囲を制限する機構を持っていない. よって, 本研究ではDHRNetのdeformable convolutionにおける適切な需要野の範囲を検討し, 受容野の広がりを制限する機構を追加することで受容野を適切な範囲に収めることを研究目的とする.		% 背景
	\begin{chapter}{提案手法}\label{chap:teian}
本研究では\ref{sec:mokuteki}節で述べたように, DHRNetのdeformable convolution部分に受容野の広がりを制限する機構を追加することで, 受容野を適切な範囲に収めることを目指す. そのために, DHRNetの構造を詳しく説明した後に, offsetの範囲に制限を設ける新たな活性化関数を提案する. 

\section{DHRNet (Deformable convolution High-Resolution Network)}\label{sub:dhrnet}
DHRNet はネットワーク全体を通して高解像度を維持するアーキテクチャである. 図\ref{fig:hrnet_zentai}に示すようにHRNetは異なる解像度で処理をするネットワークを組み合わせた4つのステージで構成されている. ステージ1は入力と同等の解像度で処理するネットワークのみで構成される. 続くステージ2ではステージ1のネットワークに, 入力の1/2倍スケールの低解像度を処理するネットワークを追加する. さらに続くステージ3では1/4倍スケール低解像度を処理するネットワークを追加し, 最後のステージ4では1/8倍スケール低解像度を処理するネットワークを追加する. また, ステージ4にはdeformable convolution が適応されている. そして解像度を下げるたびにチャンネル数は2倍になる(16, 32, 64, 128チャンネル). 以降, 図\ref{fig:hrnet_zentai}に示した本アーキテクチャの詳細について説明していく. また, 図中のReLU, Batch Normalizationについては, それぞれ, 付録\ref{hu:ac_fn}, 付録\ref{hu:bn}で説明する. 
\begin{figure}[h]
  \centering
  \includegraphics[scale=0.7]{./images/DHRNet.eps}
  \caption{DHRNetの構造}
  \label{fig:hrnet_zentai}
\end{figure}
\newpage
\subsection{Basic Block}\label{subsec:BB}
図\ref{fig:hrnet_zentai}のピンク色で示した``Basic Block''の構造を示したものが図\ref{fig:basicblock}である. このブロックはHRNetにおいて特徴抽出を担う重要な箇所である. 一般にこのような構成で層を深くすると, いわゆる``勾配消失''により, ネットワークの重みが更新できなくなる. これを解決する方法はさまざま提案されているが, HRNetでは``ResNet(Residual Network)''\cite{resnet}で提案された``残差ブロック''を用いている. 図\ref{fig:res}に示すように, 一般的なCNNは入力$x$を数段のConvolution層に通して$H(x)$を得るのに対し, 残差ブロックでは, CNNの出力$F(x)$と入力$s$との和を出力$H(x)=F(x)+x$とする. これにより, CNNでは$F(x)=H(x)-x$なる$H(x)$と入力$s$との残差を学習すればよく, 残差信号の平均値は0に近いので勾配消失を防ぐ効果が期待できる.

\begin{figure}[H]
  \centering
  \includegraphics[scale=1]{./images/BasicBlock.eps}
  \caption{Basic Block}
  \label{fig:basicblock}
\end{figure}

\begin{figure}[H]
  \centering
  \includegraphics[scale=0.30]{./images/resnet.eps}
  \caption{一般的なCNN(右), 残差ブロック(左)}
  \label{fig:res}
\end{figure}

\subsection{Bottleneck}\label{subsec:bottleneck}
図\ref{fig:hrnet_zentai}の青色で示した``Bottleneck''はBasic Blockと同様, HRNetにおいて特徴抽出を担う箇所で, 図\ref{fig:bot}(右)に示すような構造である. 構造は残差ブロック(図\ref{fig:bot}(左))に似ているが, Bottleneckでは入力のチャンネル数を$1\times1$畳み込み(付録\ref{hu:1x1})を用いて小さくしてから$3\times3$畳み込みで特徴抽出し, 最後の$1\times1$畳み込みでチャンネル数を復元する. これにより計算コストを削減できる. 例えば, 図\ref{fig:bot}において残差ブロックの入力チャンネル64, Bottleneckが入力チャンネル数256とBottleneckの方がチャンネル数が大きいが, これらの計算コストは同等である.
\begin{figure}[H]
  \centering
  \includegraphics[scale=0.30]{./images/bottleneck.eps}
  \caption{残差ブロック(右), Bottleneck(左)}
  \label{fig:bot}
\end{figure}

\subsection{Exchange Unit}
HRNetでは異なる解像度の特徴マップ間で情報共有するために``Exchange Unit''がある. 図\ref{fig:hrnet_zentai}ではオレンジ色の部分がそれにあたる. 具体的に, Exchage Unitでは異なる解像度の特徴マップ間でバイリニア補間によるアップサンプリング, またはストライド2の$3\times3$畳み込みによるダウンサンプリングで特徴マップのサイズを合わせ, 特徴マップ同士を加算する. これにより異なる解像度の特徴マップ間での情報共有を可能にしている. 

\subsection{Up sampling}
図\ref{fig:hrnet_zentai}の紫色で示したUp samplingは1/2, 1/4, 1/8の低解像度の特徴マップをバイリニア補間でアップサンプリングし, それらをチャンネル方向を軸にして1/1解像度の特徴マップに連結する. このUp samplingブロックで得られた特徴マップを分類に用いる.

\begin{figure}[H]
  \centering
  \includegraphics[scale=0.35]{./images/exchange.eps}
  \caption{exchange unit\cite{hrnet}}
  \label{fig:exchange}
\end{figure}

\subsection{Deformable Basic Block}\label{subsec:DBB}
図\ref{fig:hrnet_zentai}の赤色で示した``Deformable Basic Block''の構造を示したものが図\ref{fig:deformablebasicblock}である. 
``Deformable Basic Block''は, Basic Block の入力から数えて1番目と3番目の$3\times3$Convolutionをdeformable convolution に変更した

\begin{figure}[H]
  \centering
  \includegraphics[scale=1]{./images/DeformBlock_deform-conv2.eps}
  \caption{Defromable Basic Block}
  \label{fig:deformablebasicblock}
\end{figure}


\section{Deformable Convolution}
deformable convolution畳み込みのサンプリング位置からの変位である``offset''を計算, 学習をしてフィルタのサンプリング位置を動的に変化させる. そのため画像中の物体のスケール, 形状に合わせて受容野のサイズ, 形状を変えることが可能である.\\
\begin{figure}[H]
  \centering
  \includegraphics[scale=0.50]{./images/Defconv_offset.eps}
  \caption{Deformable Convolution}
  \label{fig:defconv_offset}
\end{figure}
畳み込みのサンプリング位置$p_n$の集合を$\mathcal{R}$, $w$を重み, $x$を入力特徴マップ, $y$を出力特徴マップと表したとき, 出力特徴量マップ$y$の各位置$p_0$に対応するConvolutionは式(\ref{eq:conv})のように書ける.
\begin{eqnarray}
  R = \{(-1,-1), (-1,0),\cdots,(0,1), (1,1)\} \nonumber \\
  y(p_0) = \sum_{p_n\in R}^{} w(p_n)x(p_0 + p_n) \label{eq:conv} 
\end{eqnarray}
deformable convolutionでは, 式(\ref{eq:def})に示すようにconvolutionのサンプリング位置$p_0 + p_n$にoffset $\Delta p$を加算することで受容野を可変にしている.
\begin{equation}
y(p_0) = \sum_{p_n\in R}^{} w(p_n)x(p_0 + p_n + \Delta p_n) \label{eq:def}
\end{equation}


\subsection{offset}
offsetの算出方法を図\ref{fig:defconv_offset2}に示す. 
offsetの算出では, まず入力特徴マップを図\ref{fig:defconv_offset2}中の緑の枠線で示す追加のconvolutionに通すことで, 緑の立方体で示す入力特徴量の全てのサンプリング点のoffsetであるoffset fieldを出力する. offsetは$x$方向, $y$方向の移動量を示す2次元ベクトルであるため, offset fieldのチャンネル数は$2N$チャンネルである($N$はフィルタカーネルのサイズで, $3\times3$フィルタカーネルの場合$N=9$). 次にoffset field から 各サンプリング点のoffsetを取り出すことで,  図\ref{fig:defconv_offset2}中の緑の格子て示すoffsetを決定する. そして, 図\ref{fig:defconv_offset2}中の青い点線で示すように, offsetでconvolutionの受容野を変形する. しかし, offsetは整数であるとは限らないため, offsetによって変化した後のサンプリング位置が非整数座標になってしまう場合がある. その場合, 非整数座標に画素値は存在しないため, バイリニア補間で非整数座標の画素値を算出する必要がある(式(\ref{eq:bilini})).($G$はバイリニア補間, $q$は$p_0 + p_n + \Delta p_n$の周辺の整数座標) 
\begin{equation}
x(p_0 + p_n + \Delta p_n) = \sum_{q}^{} G(q, p_0 + p_n + \Delta p_n)x(q) \label{eq:bilini}
\end{equation}
offset $\Delta p_n$を学習するために, 式(\ref{eq:def}), (\ref{eq:bilini})のバイリニア補間をもとに勾配を求めると式(\ref{eq:bilini_back})のように求まる. これを用いて, 誤差逆伝播法(付録\ref{hu:backprop})により学習を行う.
\begin{eqnarray}
  \frac{\partial y(p_0)}{\partial \Delta p_n} &=& \sum_{p_n\in R}^{} w(p_n)\frac{\partial x(p_0 + p_n + \Delta p_n)}{\partial \Delta p_n} \nonumber \\
  &=& \sum_{p_n\in R}^{} [w(p_n) \sum_{q}^{} \frac{\partial G(q, p_0 + p_n + \Delta p_n)}{\partial \Delta p_n}x(q)]
  \label{eq:bilini_back}
\end{eqnarray}

\begin{figure}[H]
  \centering
  \includegraphics[scale=0.4]{./images/defconv_offset2.eps}
  \caption{Deformable Convolutionの流れ\cite{defconv}}
  \label{fig:defconv_offset2}
\end{figure}

\section{受容野の範囲に制限を設ける活性化関数の提案}\label{sec:teian_ac_fn}
deformable convolutionのoffsetによって決定する受容野が収まるべき範囲は, 畳み込み層を重ねたときの, ある入力画像の画素が影響を及ぼす範囲によって決定できると考えた. 例えばストライド1, padding1の$3\times3$畳み込みの場合は, 1つの画素が最大で$3\times3$の範囲の出力特徴量に影響を及ぼす. 図\ref{fig:conv_ptn}に, ストライド1, padding1の$3\times3$畳み込みにより入力特徴量の影響を及ぼす範囲が広がる例を示す. 図\ref{fig:conv_ptn}中にある入力データの灰色はpaddingであり, 白はpadding前の元の入力データである. そして, 入力データ内の赤色で示す特徴量を入力として, $3\times3$畳み込みにより得られた出力特徴量が出力特徴量マップの薄い赤色部分である. さらに, downsamplingの際にも各特徴量が影響を及ぼす範囲は広がる. この入力特徴量が影響を及ぼす範囲内に受容野が収まれば, offsetにより変化したサンプリング点に変化前のサンプリング点の特徴が必ず含まれるようになる. \\
\begin{figure}[H]
  \centering
  \includegraphics[scale=1.5]{./images/conv_ptn.eps}
  \caption{ストライド1, padding1の$3\times3$畳み込み層により特徴量の影響を及ぼす範囲が広がる例}
  \label{fig:conv_ptn}
\end{figure}
しかし, 現状のdeformable convolutionのoffsetは図\ref{fig:conv_ptn}中の緑色の枠に示す畳み込み層の出力であり, 値の範囲を制限をせずに受容野の変更に用いている. そのため, deformable convolutionは入力特徴量が影響を及ぼす範囲外まで受容野を拡大することが可能な構造をしている. そこで, 受容野の範囲を特徴量の影響を及ぼす範囲に制限するために, DHRNetの低解像度部分における特徴量が影響を及ぼす範囲を確認した. 

\subsection{DHRNetにおける特徴量の広がり}\label{subsec:dhrnet_feature_map}
図\ref{fig:fig:hrnet_zentai}より, DHRNetの入力画像はまずストライド2, padding1の$3\times3$畳み込み層を2層通る. このストライド2, padding1の$3\times3$畳み込み層では, 出力特徴マップのサイズが入力特徴マップの1/2になるのに加え, ある入力特徴量が影響を及ぼす範囲が出力特徴マップの$2\times2$の範囲まで広がる. 図\ref{fig:3x3_s2_p1}は, $8\times8$の入力特徴マップをストライド2, padding1の$3\times3$畳み込み層で特徴抽出したときに特徴量の影響を及ぼす範囲が広がる例である. 図\ref{fig:3x3_s2_p1}の入力画像の赤色で示す入力画素は, 出力特徴量マップの薄い赤色で示す$2\times2$の範囲の算出に用いられる. そして, 2層目の畳み込み層では特徴マップのサイズは1/2になるが, ある画素が影響を及ぼす範囲は$2\times2$のままである. 

\begin{figure}[H]
  \centering
  \includegraphics[scale=0.1]{./images/3x3_s2_p1.eps}
  \caption{ストライド2, padding1の$3\times3$畳み込み層により特徴量の影響を及ぼす範囲が広がる例}
  \label{fig:3x3_s2_p1}
\end{figure}

その後, bottleneckで$1\times1$, $3\times3$, $1\times1$の畳み込み層を通る. $1\times1$畳み込み層はチャンネル方向のみ畳み込むため, 特徴量の影響を及ぼす範囲は変化しない. botleneckの$3\times3$畳み込みはストライド1, padding1であるため, 図\ref{fig:3x3_s1_p1}の赤色で示す$2\times2$の範囲の特徴量が出力特徴マップの薄い赤色で示す$4\times4$の範囲まで広がる. 

\begin{figure}[H]
  \centering
  \includegraphics[scale=0.1]{./images/3x3_s1_p1.eps}
  \caption{入力特徴マップの$2\times2$の特徴量が影響を及ぼす範囲が, ストライド1, padding1の$3\times3$畳み込み層により広がる例}
  \label{fig:3x3_s1_p1}
\end{figure}

このようにDHRNetでのある入力画素の影響を及ぼす範囲を計算したところ, 低解像度部分の特徴マップでは$13\times13$の範囲まで広がっていた. ここで, deformable convolutionのoffsetを算出する畳込み層のカーネルサイズは$3\times3$であるため, offsetには$15\times15$の範囲の特徴量が影響を及ぼす. つまり, offsetの範囲は上下左右7画素の範囲に制限するべきである. 

\subsection{Hard tanh7}\label{subsec:hard_tanh7}
本研究では,算出されたoffsetの範囲を制限するhard tanh7をoffsetに適応することを提案する. 提案するhard tanh7は, hard tanh\cite{raghu2017expressive}を拡張した活性化関数である.  hard tanh関数は図\ref{fig:hardtanh}に示すように, 入力値が-1より小さい場合と1より大きい場合は出力値が常に0であり, -1以上, 1以内の範囲では入力値をそのまま出力する活性化関数であるがdeformable convolution で制限したい値域とは異なる. 

% 入力特徴量が影響を及ぼす範囲外まで受容野が拡大すると, 入力特徴量とは関係のない位置の特徴量をdeformable convolutionの特徴抽出に用いることになるため, 学習が進まないと考えた.
\begin{figure}[H]
  \centering
  \includegraphics[scale=0.4]{./images/hardtanh.eps}
  \caption{hard tanh関数}
  \label{fig:hardtanh}
\end{figure}

そのため, 提案するhard tanh7は図\ref{fig:hardtanh6}に示すように, hard tanhの値域を-7以上, 7以下に拡張した. 
\begin{figure}[H]
  \centering
  \includegraphics[scale=0.45]{./images/hardtanh6.eps}
  \caption{hard tanh7関数}
  \label{fig:hardtanh6}
\end{figure}
ここで, harh tanh7の入力を$x$としたときの式を式(\ref{eq:hard_tanh7})に示す. 入力$x$が$x<-7, x>7$ならば出力値$hardtanh7(x) =0$であり, $-7<=x<=7$の範囲内ならば$hardtanh7(x)=x$である. 
\begin{equation}
  hardtanh7(x) = 
      \begin{cases}
        7  (x>7)\\ 
        x  (-7<=x<=7)\\
        -7  (x<-7) \label{eq:hard_tanh7}
      \end{cases}
\end{equation}
この提案する活性化関数にoffsetを通すことで, offsetを特徴量が影響を及ぼす範囲内で学習させる.

\section{offsetの中心画素を固定する提案}\label{sec:teian_c0}
現状のアーキテクチャでは畳み込みの中心サンプリング点も移動しており, 入力特徴量の中でサンプリングされない特徴が出てきてしまう可能性がある. 図\ref{fig:deform_sampling_pg}はpadding1, $5\times5$サイズの入力特徴マップにおけるdeformable convolutionのサンプリング位置の例を示す. 図\ref{fig:deform_sampling_pg}の左に示す入力特徴量を, 中央の緑色で表す可変形状の畳み込みカーネルで畳み込む場合, 右に示すように特徴マップでサンプリングに用いない箇所が存在する. それに加え, 畳み込みの位置普遍性が失われてしまう問題もあるため, 入力特徴量において特徴抽出に用いられない点があることは問題であると考えた.
\begin{figure}[H]
  \centering
  \includegraphics[scale=0.05]{./images/deform_sampling_pg.eps}
  \caption{deformable convolutionにおけるサンプリング位置}
  \label{fig:deform_sampling_pg}
\end{figure}
そこで, 本研究では全ての入力特徴量を特徴抽出に使用するために, deformable convolutionの中心画素のoffsetを0に固定することを提案する. 図\ref{fig:deform_c0_sampling_pg}の中央に, deformable convolutionの中心画素のoffsetを0に固定した可変形状の畳み込みカーネルを示す. 畳み込みカーネルの中心サンプリング位置は強調するために赤色で表す. 中心画素のoffsetを0に固定することで, 図\ref{fig:deform_c0_sampling_pg}の右に示すように畳み込みカーネルの中心サンプリング点で特徴マップのすべて特徴量をサンプリングすることができる.


% padding1の$5\times5$サイズの入力特徴と, $3\times3$畳み込みカーネルの中心位置のみを示した図である. 図\ref{fig:defconv_c0_offset}の上段にはある緑色は$3\times3$畳み込みカーネルの中心位置を表しており, 青い矢印で畳み込みカーネルがスライドする位置を示す. そして, 図\ref{fig:defconv_c0_offset}の下段の緑部分は, $3\times3$畳み込みカーネルの中心のサンプリング点でたたみ込まれた入力特徴量の位置を示す. 下段を見ると全ての入力特徴量が特徴抽出に用いられていることがわかる.

\begin{figure}[H]
  \centering
  \includegraphics[scale=0.05]{./images/deform_c0_sampling_pg.eps}
  \caption{中心画素のoffsetを0に固定したdeformable convolutionにおけるサンプリング位置}
  \label{fig:defconv_c0_sampling_pg}
\end{figure}

\end{chapter}
		% 提案
	\chapter{実験} \label{sec_experience}

\section{実験条件}
  本節では実験条件として, 使用するデータセット, 評価指標, ハイパーパラメータ, 実験環境について述べる. 
\subsection{データセット}
本実験では``Cityscapes''\cite{cityscapes}を使用した. Cityscapesは都市のストリートシーンの画像で構成されており, 学習用画像が2975枚, 検証用画像が500枚, テスト用画像が1525枚の計5000枚用意されている. 学習の対象とするクラスはroad, sidewalk, building, wall, fence, pole, traffic light, traffic sign, vegetation, terrain, sky, person, rider, car, truck, bus, train, motorcycle, bicycleの19クラスである. 

\subsection{評価指標}
学習したモデルの性能を, スレットスコア(Threat Score)を用いた以下の2つの指標で評価する.
\begin{itemize}
\item IoU (Intersection over Union)
\item MIoU (Mean Intersection over union)
\end{itemize}
スレットスコアは, 正解データと予測結果を2値で比較し, その正誤を表\ref{tab:hixyouka}のような4つの状態で表すConfusion Matrixを用いて定義する.
\begin{table}[H]
  \centering
  \caption{Confusion Matrix}
  \begin{tabular}{c|c|c}\hline \hline
     & 正解はA & 正解はAではない \\ \hline 
    Aと予測 & TP(True Positive) & FP(False Positive) \\ \hline
    Aではないと予測 & FN(False Negative) & TN(True Negative) \\ \hline \hline
  \end{tabular}
  \label{tab:hixyouka}
\end{table}
True, Falseは正しく予測できたかを表し, Positive, Negativeは正と負のどちらに予測されたかを表している. semantic segmentationおいて, True Positiveは例えばAというクラスを正しくAと判定できたピクセル数, False Positiveは別のクラスであるにもかかわらずクラスAと誤った判定したピクセル数, False NegativeはAというクラスにもかかわらず別のクラスと誤った判定をしたピクセル数である. \\
次に, スレットスコアの計算式を示す. そもそもスレットスコアは気象分野で用いられる指標だが, ここでTP, FPをそれぞれTrue Positive, False Positiveのピクセル数とし, FNをFalse Negativeのピクセル数として, semantic segmentationの評価値としてスレットスコアを用いたとき, このスコア値をIoU(Intersection over Union)と呼び, 式(\ref{eq:iou})のように書ける. IoUはクラスごとに求まるが, これを全体のクラス数$C$で平均したものをMIoU(Mean Intersection over Union)と呼び, 式(\ref{eq:miou})のように書ける.

\begin{eqnarray}
  \rm{IoU} &=& \frac{\rm{TP}}{\rm{TP} + \rm{FP} + \rm{FN}}\label{eq:iou}\\
  \rm{MIoU} &=& \frac{1}{C} \sum_{C=1}^C \frac{\rm{TP}_c}{\rm{TP}_c + \rm{FP}_c + \rm{FN}_c}\label{eq:miou}
\end{eqnarray}

\subsection{ハイパーパラメタ}\label{sub:hypara}
従来手法であるHRNet\cite{hrnet}と比較するため, 同じハイパーパラメタを使用した. 使用したハイパーパラメタを表\ref{tab:hypara}に示す. 
\begin{table}[H]
  \centering
  \caption{使用したハイパーパラメタ}
  \begin{tabular}{c|c} \hline \hline
     Batch Size & 12 \\ \hline 
     Epoch & 484  \\ \hline
     活性化関数(中間層) & ReLU \\ \hline
     活性化関数(出力層) & SoftMax\\ \hline \hline
  \end{tabular}
  \label{tab:hypara}
\end{table}

\subsection{実験環境}
本研究では以下の環境で実験を行った.
\begin{itemize}
\item Python 3.6.12
\item Pytorch 1.5
\item Tesla V100(32GB)
\end{itemize}


\section{実験結果及び検討}
% 提案手法の有用性を確認するために2つの実験を行う. なお, 
本実験の目的の1つがdeformable convolutionのoffsetベクトルの挙動を評価することである. しかし, deformable convolutionが2つ含まれている既存のDHNetでは, offsetベクトルの挙動を複合的に考える必要があるため, 妥当な評価ができないと考えた.
よって実験では, DHRNetのDeformable Basic Blockを 図\ref{fig:defbasicblockv2}に示すようなdeformable convolution を1層のみに変更したDHRNet(deform-conv 1層)をベースにして実験を行う.

\begin{figure}[H]
    \centering
    \includegraphics[scale=1]{./images/DeformBlock_deform-conv1.eps}
    \caption{Deformable Basic Block(deform-conv 1層)}
    \label{fig:defbasicblockv2}
\end{figure}

\subsection{DHRNetの有効性の確認}
本研究では, 提案する活性化関数をoffsetに適応することと, 中心画素のoffsetを0に固定することとの妥当性を検討する. 
そのために, まずDHRNet(deform-conv 1層)の有効性を確認する. これを実験1とする. HRNetとDHRNet(deform-conv 2層), DHRNet(deform-conv 1層)のクラス全体の結果を表\ref{tab:tekiyouiti}に示す.

\begin{table}[H]
    \centering
    \caption{実験1の結果(クラス全体)}
    \begin{tabular}{c|c} \hline \hline
       & MIoU(\%)  \\ \hline
      HRNet & 70.26  \\ \hline
      DHRNet(deform-conv 2層) & $\mathbf{74.22}$  \\ \hline
      DHRNet(deform-conv 1層) & 73.52  \\ \hline \hline 
    \end{tabular}
    \label{tab:tekiyouiti}
\end{table}
表\ref{tab:tekiyouiti}より, DHRnet(deform-conv)はDHRNetより0.7pt減少したが, HRNetからは3.26pt向上した. \\
次に, \ref{sec:mokuteki}節で示した (1)受容野の形状が物体によって変化しているように見えない. (2)受容野が影響を及ぼす範囲外まで広がっている.  という問題点がDHRNet(deform 1層)にも現れているかを確認するため, offsetの可視化を図\ref{fig:def1_4}に, offsetのx軸成分とy軸成分の分布を表したグラフを\ref{fig:def1_4_x}と\ref{fig:def1_4_y}に示す. \ref{fig:def1_4_x}と\ref{fig:def1_4_y}のx軸はoffsetの値であり, y軸は頻度である. なお, 赤い点線はoffsetベクトルの存在するべき範囲を表す.

\begin{figure}[H]
    \begin{center}
    \includegraphics[scale=0.15]{./images/deform1ly4.eps}
    \end{center}
    \caption{DHRNetのoffsetの可視化}
    \label{fig:def1_4}
\end{figure}

可視化結果(図\ref{fig:def1_4})の評価画素ごとに受容野の形状を比較すると大きさによる変化は確認できるが, 形状の変化はあまり見られない. 次に\ref{fig:def1_4_x}と\ref{fig:def1_4_y}を見ると, -6以上6以下の範囲外に値が存在する. よって, DHRNet(deform-conv 1層)を実験に使用するのに適しているといえる.
% また, 図\ref{fig:def1_4}の各評価画素の中心サンプリング点が

\begin{figure}[H]
  \begin{center}
  \includegraphics[scale=0.20]{./images/deform1_4_0_x.eps}
  \end{center}
  \caption{offsetのx成分の分布}
  \label{fig:def1_4_x}
\end{figure}

\begin{figure}[H]
  \begin{center}
  \includegraphics[scale=0.20]{./images/deform1_4_0_y.eps}
  \end{center}
  \caption{offsetのy成分の分布}
  \label{fig:def1_4_y}
\end{figure}


\subsection{提案手法の有効性の確認}
実験1でDHRNet(deform-conv 1層)が評価実験のベースとできることを確認できたため, DHRNet(deform-conv 1層)を用いて提案手法の評価を行う. これを実験2とする. 本研究では, offsetの範囲に制限を設ける活性化関数を追加することと, 中心画素のoffsetを0に固定することとの2つ提案をした.
仮説としては, offsetの範囲に制限を設けることにより, 学習が促進され, 中心画素のoffsetを0に制限することでさらに学習が進むと考えた. この仮説より, 3つの条件で実験を行った.

\begin{description}
   \item[提案手法1] DHRNet(deform-conv 1層)のoffset算出部分にhard tanh7を導入
   \item[提案手法2] DHRNet(deform-conv 1層)の中心画素のoffsetを0に固定
   \item[提案手法3] DHRNet(deform-conv 1層)のoffset算出部分にhard tanh7を導入することに加え, 中心画素のoffsetベクトルを0に固定
\end{description}

実験2のクラス全体の結果を表\ref{tab:ikken2_miou}, クラスごと結果を表\ref{tab:zikken2_iou}に示す.

\begin{table}[H]
    \centering
    \caption{実験2の結果(クラス全体)}
    \begin{tabular}{c|c|c} \hline \hline
      & baseline & MIoU(\%)  \\ \hline
      & HRNet & 70.26  \\ \hline
      deform-conv 1層 & DHRNet & 73.52  \\ \hline
      \begin{tabular}{l}deform-conv 1層 \\+ hard hanh7\end{tabular} & DHRNet & 73.13  \\ \hline
      \begin{tabular}{l}deform-conv 1層\\(中心のoffset 0)\end{tabular} & DHRNet & 73.91  \\ \hline
      \begin{tabular}{l}deform-conv 1層 \\+ hard tanh7\\(中心のoffset 0)\end{tabular} & DHRNet & $\mathbf{74.09}$  \\ \hline \hline
    \end{tabular}
    \label{tab:ikken2_miou}
\end{table}

\begin{table}[H]
  \centering
  \caption{実験2の結果(クラスごと)}
  \scalebox{0.6}{
  \begin{tabular}{c|c|c|c|c} \hline \hline
   クラス & \multicolumn{4}{|c}{IoU(\%)}\\ \hline 
   & HRNet & deform-conv 1層 & \begin{tabular}{l} deform-conv 1層 \\ + hard tanh7\end{tabular} & \begin{tabular}{l} deform-conv 1層 \\(中心のoffset 0)\end{tabular} & \begin{tabular}{l} deform-conv 1層\\ + hard tanh7 \\(中心のoffset 0)\end{tabular} \\ \hline 
    road & 97.64 & 97.88 & 97.88 & $\mathbf{97.90}$ & 97.89  \\ \hline
    sidewalk & 82.21 & 83.00 & 83.12 & $\mathbf{83.43}$ & 83.17 \\ \hline
    building & 90.90 & 91.62 & 91.55 & 91.59 & $\mathbf{91.66}$ \\ \hline
    wall & 47.10 & 51.90 & 51.39 & 52.72 & $\mathbf{53.42}$ \\ \hline
    fence & 53.05 & 55.40 & 55.33 & 55.91 & $\mathbf{56.08}$ \\ \hline
    pole & 60.42 & 61.67 & 61.77 & 61.81 & $\mathbf{62.16}$ \\ \hline
    traffic light & 63.87 & 63.17 & 64.48 & 64.63 & $\mathbf{64.95}$ \\ \hline
    traffic sign & 74.33 & 75.27 & $\mathbf{75.44}$ & 75.43 & 75.02 \\ \hline
    vegetation & 91.89 &  91.97 & 92.10 & 92.12 & $\mathbf{92.20}$ \\ \hline
    terrain & 62.19 &  62.52 & 62.35 & $\mathbf{63.26}$ & 62.80 \\ \hline
    sky & 93.74 & $\mathbf{94.40}$ & 94.34 & 94.28 & 94.26 \\ \hline
    person & 78.58 & 79.00 & 79.35 & $\mathbf{79.47}$ & 79.27 \\ \hline
    rider & 53.95 &  57.44 & 57.02 & $\mathbf{58.73}$ & 57.39 \\ \hline
    car & 93.14 & $\mathbf{93.84}$ & 93.80 & 93.77 & 93.81 \\ \hline
    truck & 50.54 & $\mathbf{67.19}$ & 59.68 & 63.53 & 66.41 \\ \hline
    bus & 70.68 & 80.93 & 77.22 & 80.10 & $\mathbf{82.53}$ \\ \hline
    train & 47.08 & 61.85 & 63.04 & $\mathbf{66.42}$ & 65.90 \\ \hline
    motorcycle & 49.82 & 54.48 & $\mathbf{55.75}$ & 55.61 & 55.16 \\ \hline
    bicycle & 73.73 & 73.57 & $\mathbf{73.90}$ & 73.48 & 73.58 \\ \hline \hline
  \end{tabular}
  }
  \label{tab:zikken2_iou}
\end{table}

表\ref{tab:ikken2_miou}より, DHRNet(deform-conv 1層)にhard tanh6を適応した提案手法1は, DHRNet\\(deform-conv 1層)から0.33pt向上している. そしてはDHRNet(deform-conv 1層)にhard tanh6を適応し, 中心のoffsetを0に固定した提案手法2は, DHRNet(deform-conv 1層)から0.24pt低下した. また\ref{tab:ikken2_miou}より, hard tanh6を適応した提案手法1は, DHRNet(deform-conv 1層)と比べて19クラス中13クラスで精度が高かった. よって, deformable convolutionのoffsetの範囲に制限を設けることで学習が促進することがわかった. 
次にoffsetに制限を加えたことでどのように受容野が変更されているかを確認するために, offsetの可視化結果を示す.

\begin{figure}[H]
    \centering
    \includegraphics[scale=0.15]{./images/deform_1_4_tanh6.eps}
    \caption{deform-conv+hard tanh6のoffset}
    \label{fig:deformtanh6}
\end{figure}
\begin{figure}[H]
    \centering
    \includegraphics[scale=0.15]{./images/deform_1_4_tanh6_c0.eps}
    \caption{deform-conv+hard tanh6(中心のoffsetベクトルを0に固定)のoffset}
    \label{fig:deformtanh6c0}
\end{figure}

図\ref{fig:deformtanh6}と図\ref{fig:deformtanh6c0}より, 受容野が指定した範囲内に制限されていることが確認できた.		% 実験
	\chapter{結論}\label{sec_conc}
本研究では受容やの大きさや形状が可変な deformable convolution が物体によって受容野の形状をどのように変更しているのかを受容やを可視化することにより視覚的に評価した. その結果, 受容野が学習できる範囲外までの広がっていると考えた. そこで受容野の広がる範囲をoffsetを算出する特徴マップが表現できる範囲内に抑えるために新たな活性化関数を提案した. 提案した活性化関数をMIoU, IoU で評価するのに加え, 受容野を可視化して意味のある変形をしているかを確認した. 結果は, MIoUが0.3pt向上し, IoUは多くのクラスにおいて精度の向上をパラメータの追加をせずに達成した. よってdeformable convolutionの受容野を制限することで学習が促進すると結論付けられる.		% 結論

	\acknowledgement{謝辞}

本論文執筆にあたり,貴重な時間を割き熱心にご指導頂いた荒井秀一教授に心から感謝します.また自らの研究が忙しい中,貴重な時間を割いて助言をして下さった知識情報処理研究室の先輩を始め,1年間苦楽を共にした知識情報処理研究室の皆様に感謝します.

\begin{flushright}

	\large

	2022年 年01月24日

	\Large

	折田 汐凪

\end{flushright}
			% 謝辞
	\nocite{*}
	\bibliography{ref}
	\appendix
\begin{chapter}{基礎事項}
\section{活性化関数}\label{hu:ac_fn}
活性化関数とは, 線形分離できない問題を解くために, ニューラルネットワークの各ユニットからの出力を正規化する非線形な関数である.代表的な関数として Sigmoid 関数や Tanh 関数などがあるが, Semantic Segmentationのネットワークの場合, 中間層では``ReLU''\cite{relu}, 出力層では分類問題で多く用いられる``Softmax''\cite{softmax}を使用するのが一般的である. 以下に本研究で使用する活性化関数について記述する.  
\subsection*{Softmax 関数}
Softmax 関数\cite{softmax}は主に多クラス分類問題での出力層に用いられる.出力層のニューロンの数が $n$ 個あるとして, $k$ 番目の入力 $x_k$ に対する出力 $y_k$ を求める計算式は, 
\begin{equation}
    \begin{split}
        y_k = \frac{\exp(a_i)}{\sum_j^n \exp(a_j)},\;(i=1,\ldots,n)
    \end{split}
\end{equation}
となる.Softmax 関数の出力は, 0 から 1 の間の実数になる.また, Softmax 関数の出力 の総和は 1 になる.この性質から Softmax 関数の出力は多クラス問題における確率値とし て解釈できる.
\subsection*{ReLU(Rectified Linear Unit) 関数}
ReLU 関数\cite{relu}は入力値が正ならば値をそのまま出力し, 負ならば 0 を出力する関数である.
入力を x としたときの式は, 
\begin{equation}
h(x) = 
    \begin{cases}
        x  (x>0)\\
        0  (x<=0)
    \end{cases}
\end{equation}
となる.このように ReLU 関数はシンプルであることから, 層の数が多くなっても安定した学習ができるため, 深層学習の研究において最もよく用いられている活性化関数である.

\subsection*{tanh(Hyperbolic Tangent) 関数}
tanh関数は式(\ref{eq:tanh})に示す関数であり, 出力は-1から1の範囲である.
\begin{equation}
\tanh(x) = \frac{e^{x}-e^{-x}}{e^{x}+e^{-x}} \label{eq:tanh}
\end{equation}

\subsection*{hard tanh 関数}
hard tanh関数は式(\ref{eq:hard_tanh})に示す関数であり, tanhよりも計算量が少ない. 
\begin{equation}
  hard tanh(x) = 
      \begin{cases}
        1  (x>1)\\ 
        x  (-1<=x<=1)\\
        0  (x<-1) \label{eq:hard_tanh}
      \end{cases}
\end{equation}

\section{Batch Normalization}\label{hu:bn}
``Batch Normalization''\cite{batchnorm}は, データ全体をいくつかの小さなデータの集合に分割した``ミニバッチ''ごとにデータの分布を平均を0, 分散が1になるように正規化する手法である. この手法により, 学習データに過度に適応してしまう``過学習''を抑制することができる.
\section{バイリニア補間}\label{hu:bi}
バイリニア補間は画像の拡大, 回転, 変形を行うときの画素補間の一種である. 求めたい座標の画素値の周囲$2\times2$画素値を参照し, その加重平均値を用いて補間する. バイリニア補間の式を式(\ref{eq:bi})に示す. なお$f$は画素値, $(x_0, y_0)$は着目点の座標, $(x, y)$は$(x_0, y_0)$の周囲の格子点座標を表す.
\begin{equation}
  \begin{split}
  f(x_0, y_0) &= f(x, y)(1-\alpha)(1-\beta)+f(x+1, y)\alpha(1-\beta)\\
  &+ f(x, y+1)(1-\alpha)\beta+f(x+1, y+1)\alpha\beta\\ \label{eq:bi}
  \end{split}
\end{equation}
\begin{equation}
  \begin{split}
  x &= \lfloor x_0 \rfloor  \nonumber\\
  y &= \lfloor y_0 \rfloor \nonumber\\
  \alpha &= x - \lfloor x_0 \rfloor  \nonumber\\
  \beta &= y -\lfloor y_0 \rfloor  \nonumber
  \end{split}
\end{equation}
 
\section{$1\times1$畳み込み(Pointwise Convolution)}\label{hu:1x1}
$1\times1$畳み込み(Pointwise Convolution)は, $1\times1$カーネルで畳み込み演算を行う. 通常のConvolutionは入力特徴マップの空間方向(幅, 高さ)とチャンネル方向に同時に畳み込む. 対し, $1\times1$Convolutionでは, チャンネル方向のみに畳み込む. そのため$1\times1$Convolutionはチャンネル数の調整に用いられる.  


\section{誤差逆伝播法(Back-propagation)}\label{hu:backprop}
ニューラルネットワークは損失関数(付録\ref{hu:lossfunc})が最小になるように学習していくが, その際に損失関数が減少する方向を示す勾配が必要になる. この勾配を効率よく求める方法として誤差逆伝播法がある. 誤差逆伝播法は, 誤差を出力層から前の層へ次々と伝播していくことで, 各層の重みの勾配を求める. 

\section{損失関数}\label{hu:lossfunc}
損失関数はネットワークの出力結果と正解データの誤差を表す関数である. 出力結果が正解データに近づくほど損失関数から得られる誤差は小さく, 出力結果が正解データから離れるほど誤差は大きくなる. ニューラルネットワークでは損失関数が最小になるように学習する. 


\end{chapter}
\end{document}

%--------------------------------------------------
